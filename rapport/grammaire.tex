%commande \alt permet de mettre une | en début de ligne pour continuer une expression en mettant un retour à la ligne
%grammarC permet de centrer le code le paramètre contient la plus longue ligne

	\begin{grammarC}{<expression> ::= <expression-ent> | <opérateur> <entiers>}
		
		<expression> ::= <procédure>
		
		<procédure> ::= "(" <opérateur> <données> ")"
				
		<opérateur> ::= "+" | "-" | "*" | "=" | "<" 
		
		<données> ::= <donnée> <données> | <epsilon>
		
		<donnée> ::= <valeurs> | <procédure> | $\epsilon$
		
		<valeurs> ::= <valeurNeg> | <valeurPos>
		
		<valeurNeg> ::= "-" <entier>
		
		<valeurPos> ::= <signePos> <entier>
		
		<entier> ::= <digit> <entier> | <digit>

		<signePos> ::= "+" | $\epsilon$ 
		
		<identificateur> ::= <letter> ( <identificateur> | $\epsilon$)
		
		<digit> ::= '0' | '1' | '2' | '3' | '4' | '5' | '6' | '7' | '8' | '9'
		
		<letter> ::= 'a' | 'b' | 'c' | 'd' | 'e' | 'f' | 'g' | 'h' | 'i' | 'j'
			    \alt 'k' | 'l' | 'm' | 'n' | 'o' | 'p' | 'q' | 'r' | 's' | 't'
				\alt 'u' | 'v' | 'w' | 'x' | 'y' | 'z'
		
		
	\end{grammarC}

\todo[inline]{support des constantes booléenne \#f \#t}
\todo[inline]{voir pour le support des opération plus complexe, println quotient, modulo, if et let}
\todo[inline]{Ajouter des identificateurs plus complets}
